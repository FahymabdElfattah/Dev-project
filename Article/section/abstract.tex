\begin{abstract}
	\initial{T}\textbf{he application of pruning algorithms in computer vision is explored, focusing on image stitching and photo-based 3D modeling. Pruning involves selectively removing connections and neurons from neural networks to reduce storage requirements, energy consumption, and address network complexity and overfitting. Limitations associated with pruning algorithms include potential loss of precision, sensitivity to initialization, and the need for retraining. Mobile applications can benefit from pruning algorithms, as exemplified by the MobileNets framework, which incorporates depthwise separable convolutions and pruning techniques to reduce model size and computational complexity while preserving accuracy. The complementary nature of pruning algorithms in optimizing computer vision tasks is emphasized.}
	
\end{abstract}