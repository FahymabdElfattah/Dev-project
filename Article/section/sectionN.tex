\section*{Discussion et perspectives}
\subsection*{Comparaison}
Lorsque nous comparons les modèles LeNet3, ResNet20, ResNet44, RCNN, OpenFace, Farmeback, SGBM et OpenMVG à l'algorithme de Pruning, qui est une technique de compression de modèle, pour les tâches de dématriçage (demosaic), de débruitage (denoise), de transformation (transform), de cartographie de gamme (gamut map) et de compression gamma (gamma comp), nous pouvons noter les différences significatives entre ces approches.
LeNet3, ResNet20 et ResNet44 sont des architectures de réseau de neurones convolutifs classiques, qui peuvent être soumises à l'algorithme de Pruning pour réduire leur taille et leur complexité. Cependant, leur conception initiale n'est pas spécifiquement adaptée aux tâches de dématriçage, de débruitage, de transformation, de cartographie de gamme ou de compression gamma. L'application de Pruning à ces modèles peut aider à réduire leur consommation d'énergie, mais cela ne résout pas directement ces tâches spécifiques.
RCNN, OpenFace, Farmeback, SGBM et OpenMVG sont des techniques ou des algorithmes qui ne sont pas directement liés aux tâches de dématriçage, de débruitage, de transformation, de cartographie de gamme ou de compression gamma. Ils sont plutôt utilisés pour des tâches telles que la détection d'objets, la reconnaissance faciale, le suivi de mouvements ou la reconstruction 3D.
En revanche, l'algorithme de Pruning est une technique qui peut être appliquée à différents modèles de réseau de neurones, y compris ceux conçus pour des tâches spécifiques. Ainsi, si nous disposons d'un modèle spécifique pour l'une de ces tâches (par exemple, un modèle de dématriçage ou de débruitage), nous pouvons appliquer l'algorithme de Pruning pour réduire la consommation d'énergie de ce modèle tout en conservant ses performances acceptables.
En conclusion, l'algorithme de Pruning est une technique de compression de modèle qui peut être appliquée à différents modèles de réseau de neurones, y compris LeNet3, ResNet20 et ResNet44. Cependant, les autres techniques et algorithmes mentionnés ne sont pas directement liés aux tâches spécifiques mentionnées et ne bénéficient pas directement de l'algorithme de Pruning pour résoudre ces tâches.

